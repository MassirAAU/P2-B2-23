\chapter{Konklusion}
Denne rapport dokumenterer, at den nordlige gangbro ved KMD er konstrueret korrekt i form af tilstrækkelig bæreevne i forhold til de forskellige laster, som den bliver udsat for. 

Ud fra rapporten konkluderes der, at KMDs gangbro er en konstruktion, der er konstrueret som en gitterkonstruktion. En gitterkonstruktion er en konstruktion, som er opbygget af stænger, som udgør trekanter. Konstruktionen er opbygget som en statisk ubestemt konstruktion, men for at gøre det muligt at beregne de indre kræfter, ved hjælp af knudepunktsmetoden i konstruktionen, er der indsat en fiktiv stang for at gøre konstruktionen statisk bestemt.  

Stål er et velegnet materiale til gangbroen, da stål har en meget høj trækstyrke, hvilket underflangen af gangbroen vil blive udsat for. En anden grund til, at stål er et velegnet byggemateriale er, at det kan fremstilles i en lang række forskellige kvaliteter, alt efter hvilken opgave det skal have.

I denne rapport er der blevet beregnet på fire forskellige laster, som påvirker KMDs gangbro; egen-, nytte-, vind- og snelast. De forskellige laster er blevet ført ud på 48 forskellige knudepunkter. Egenlasten er den last, som en konstruktion belaster sig selv med og er blevet beregnet ud fra de materialer gangbroen er bygget af. Nyttelasten er den last, som gangbroen udsættes for af mennesker og inventar. Nyttelasten er beregnet ud fra en standard, som passer til gangbroens kategori i forhold til hvad den anvendes til. Vindlast er den last, som vinden påvirker en konstruktion med. Vindlasten er beregnet ud fra en række standarder og normer, fundet i Eurocodes, såsom hvor høj en konstruktion er, hvilket område konstruktionen ligger i, om der står andre konstruktioner i nærheden og hvor i landet konstruktionen ligger. Vindlasten er den eneste last, der i denne rapport både påvirker det vertikale og horisontale gitter. Snelasten er den last, som sne påvirker en konstruktion med, den virker udelukkende på taget af KMDs gangbro og den kan både være jævnt og ujævnt fordelt derpå. Snelasten beregnes også ud fra information, der fås i Eurocode og ud fra, at der forekommer snedriver når konstruktionen ligger imellem to højere konstruktioner, som er tilfældet for KMDs gangbro. 

Ud fra egen-, nytte-, vind- og snelasterne er der blevet beregnet lastkombinationer hvor de forskellige laster på skift er beregnet som dominerende. Der bliver ganget partialkoefficienter på lastkombinationerne for at sørge for, at lasterne bliver overvurderet i forhold til hvor store de er i den virkelige verden. Dette gøres for at sikre en mindre risiko for kollaps, da modeller og virkelighed ikke altid stemmer overens. Laster hvor partialkoefficienter er multipliceret på kaldes regningsmæssige laster. Den største regningsmæssige last bliver herefter anvendt til eftervisning af bæreevnen. På KMDs gangbro er det den lastkombination, hvor nyttelasten er dominerende, der er størst. 


Med viden om, at gangbroen er statisk bestemt, er stangkræfterne beregnet ved anvendelse af knudepunktsmetoden. Ved brug af knudepunktsmetoden løsskæres alle knudepunkter fra hinanden og der beregnes horisontal og vertikal ligevægt for indre kræfter ved knudepunkter.  Ved beregningerne af stangkræfterne bliver det fastlagt hvilke stænger der er træk- eller trykstænger. Ud fra de beregnede indre kræfter, er der opstillet en statisk model, som efterviser, at den regningsmæssige spænding er lavere end profilens maksimale tilladte spænding.  

Ud fra den statiske dokumentation kan det konkluderes, at de regningsmæssige spændinger i profilerne er mindre end den tilladte spænding, derfor er gangbroen dimensioneret tilstrækkeligt, ud fra de antagelser der er lavet i rapporten. Den fiktive stang der blev placeret for at kunne anvende knudepunktsmetoden, og ud fra beregningen af de indre kræfter kan det konkluderes, at den fiktive stang er minimalt belastet.  
