\chapter{Problemanalyse}
Dette afsnit vil indeholde en argumentation for, hvad denne rapports grundlæggende emnevalg er samt en motivationsdel til rapporten, som er givet ud fra en undren over emnet. Til sidst vil metodevalget blive begrundet.  
 
\section{Initierende problem}
KMDs langt-spændende gangbro er en konstruktion, der er udformet som en gitterkonstruktion. Dette danner grundlag for en gennemgang af analytiske modeller og metoder, der anvendes til at bestemme konstruktionens stabilitet og dimensioner. Igennem tiderne har forkerte modeller og metoder betydet delvise eller totale kollaps for bygninger. For eksempel da Superarenaen i Ballerup kollapsede den 3. januar 2003  pga. fejldimensionering af bjælkerne og i vinteren 2009-2010 hvor der skete en sammenstyrtning af en stald pga. en dimensioneringsfejl i de bærende søjler. I dette projekt vil der forsøges at opnå en forståelse af de anvendte modeller og metoder til at eftervise en konstruktions bæreevne og hvordan den er sikret via beregningerne. Dette rejser en række interessante spørgsmål, som er opdelt i følgende punktform. 
 
 \begin{itemize}
 \item Hvad påvirker gangbroens konstruktion?
 \item Hvilke metoder anvendes ved bestemmelse af bæreevnen? 
 \item Hvordan sikres konstruktionerne i de benyttede last- og beregningsmodeller? 
 \end{itemize}
 
\section{Problemformulering}
Ved den initierende problemstilling blev der set på nogle interessante områder, som projektgruppen ønsker at eftervise med denne rapport. Dette munder ud i en konkret problemstilling, hvor metoden last-system-respons vil være oplagt at anvende til beregning eller eftervisning af konstruktionens stabilitet og dimensioner. Dette sikrer processen fra modeller og forudsætninger til virkeligheden. For at man nemmere kan tyde de opstillede problemer, opdeles det i følgende underpunkter.

\begin{itemize}
\item Hvilken type konstruktion er KMDs gangbro?
\item Hvorfor er stål egnet som byggemateriale?
\item Hvilke laster bliver gangbroen påvirket af?
\item Hvordan anvendes knudepunktsmetoden til at finde den samlede lastpåvirkning for gangbroen?
\item Er dimensionerne af profilerne i gangbroen tilstrækkelige?
\end{itemize}

\section{Problemafgrænsning}
I denne rapport er projektets fokus, hvilket grundlag der er for dimensioneringen af gangbroen. Dette munder ud i en statisk model, som består af relevante beregninger, som resulterer i hvorvidt dimensioneringen af stålprofiler er korrekt eller i så fald er overdimensioneret eller underdimensioneret. Igennem rapporten vil de relevante metoder blive brugt til at besvare problemstillingerne. Rapporten er udarbejdet, som et produkt af et P1-projekt indenfor det ingeniørvidenskabelige basisår. Dette resulterer i en tidsbegrænset proces, som har betydning for rapportens indhold og derfor er det nødvendigt at lave en afgræsning. Derfor vil nedenstående problemstillinger ikke blive taget i betragtning og vil dermed blive afgrænset fra rapporten:

\begin{itemize}
\item KMDs to hovedbygninger.
\item Aalborgs lokalplan for Sturhs Brygge.
\item Geotekniks analyse af byggegrunden. 
\item Det økonomiske aspekt indenfor byggeprojektet.
\item Gangbroens energi og indeklima.
\item KMDs begrundelse for opførelsen af gangbroen.
\item De indre vindkræfter.
\item De arkitektoniske overvejelser. 
\item De hensigtsmæssige ulykkeslaster og brandsikringen.
\end{itemize}

\section{Metodevalg}
P1-projektets metodevalg er baseret på, at rapporten vil beskrive KMDs gangbro ud fra en ingeniørs synsvinkel. Metoden behandler tre hovedpunkter, som kendes fra én metode, last-system-responsmetoden. Denne metode anvendes til overvejelser for gangbroens konstruktion. Projektet vil inddrage beregninger, som vedrører de forskellige laster. Det, som optager lasterne er gangbroens konstruktion, hvormed gitterkonstruktionen er systemet, som håndterer lasterne.  Responsen af gangbroen er herefter det interessante, hvor det kan sættes i sammenhæng med, om konstruktionen er tilstrækkelig dimensioneret. Der vil blive analyseret ud fra beregninger af de indre kræfter i gangbroen.