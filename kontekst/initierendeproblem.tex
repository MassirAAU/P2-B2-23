\chapter{Initierende problem}

\section{Initierende problem}

Geologien i et byområde har stor betydning for hvordan en by udvikler sig. Dette kan man i Aalborg se blandt andet ved virksomheders placering som Aalborg Portland. Limfjorden har også stor betydning for byudvikling, især når man ser fremadrettet hvor man forventer store vandstandsstigninger, grundet klimaændringer. Geologien er ikke det eneste der har betydning for byudvikling, urbanisering er også en stor faktor, da en tilstrømning af mennesker kræver en klar plan for hvor der skal bygges nyt, og hvor der skal bygges om. 


Kommunerne i Danmark skal have en plan for hvordan de skal udvikle sig fremadrettet. Aalborg Kommunes plan går ud på at omdanne Aalborg fra en industriby til en kompetenceby, et vidensbaseret samfund. Da Aalborg er blandt Danmarks største byer, kommer der mange nye mennesker til, dette kræver at Aalborg kommune har en plan for hvor der skal opføres nye boligområder og hvor gamle boligområder skal ombygges. Det initierende problem lyder således:


\textit{Hvordan har Aalborg udviklet sig som en by gennem tiden og kan geologien i området have betydning for byens udvikling.}


I de første kapitler af rapporten vil det initierende problem blive belyst, Aalborgs historiske og fremtidige byudvikling vil blive redegjort og analyseret, der vil blive undersøgt om geologien i området har betydning for Aalborgs udvikling og der vil komme en interessentanalyse om en eventuel tilbygning til Strøybergs Palæ.

\section{Metodeafsnit}
Dette er ikke skrevet endnu.