\chapter{Procesanalyse}


\section{Indledning}
Vi har i vores P1-projekt valgt at skrive om KMDs langspændende-gangbro, hvor vi er kommet med en detaljeret beskrivelse af gangbroen samt nogle tekniske beregninger og en analyse. Vores procesanalyse har til formål at klarlægge, hvad der har været godt og hvad vi skal gøre anderledes i de kommende projekter.

\section{Projektstyring og -planlægning}.
\subsection{Beskrivelse}

\textbf{Mindmap og problemformulering}
Vi fik udleveret et projektkatalog med forskellige forslag til projekter. Da der ikke lige ummidelbart var enighed om hvad vi ville skrive om, lavede vi en demokratisk afstemning på tavlen, Se \textbf{FIGUR X}. I bilag \ref{sec:Bilag1} kan man se vores oprindelige mindmap, hvor vi kommer med forslag til emner vi vil komme ind på i rapporten. Emnerne  var baseret på ren interesse, uden den helt store baggrundsviden om statik, lastmoddelering osv. Vi udvalgte efterfølgende de emner der var højst enighed om og skrev en initierende problemformulering ud fra dem.

\textbf{Disposition}
Vi lavede meget tidligt i projektet en disposition, for hurtigt at få et overblik over hvad vi skulle igennem i projektet. Dispositionen blev også lavet for at have en tydelig rød tråd igennem projektet. Se bilag \ref{sec:Bilag2}.

\textbf{Tidsplan}
Vi lavede tidligt i projektet en tidsplan, som vi har opdateret løbende. Tidsplanen gav os et overblik over hvor lang tid vi havde til at skrive de forskellige ting, samt hvornår de forskellige dele af rapporten skulle være færdig. I tidsplanen var der også beskrevet hvornår vi skulle rette og sende noget til vejleder, smat hvornår vi havde møder med ham. Tidsplanen findes i bilag \ref{sec:Bilag3}.

\textbf{Møder}
I starten af projektet blev vi enige om at vi ville afholde korte statusmøder fra tid til anden, hvor vi vil snakke om hvor langt vi var kommet med de forskellige emner, om vi overholdte tidsplanen og om man skulle have hjælp til at skrive emnet færdigt. Vi afholder også møder med vejleder, et referat her fra ses på bilag \ref{sec:Bilag4}.

\textbf{Logbog}
I starten af projektet blev vi enig eom at vi ville skrive en kort logbog hver gang vi mødes for at arbejde. I logbogen beskriver vi hvad vi har lavet i løbet af dagen, hvilke overvejelser vi har lavet, begrundelser for overvejelserne, hvem der er ordstyrer, hvem der er referant og hvis der er sket noget særligt.

\subsection{Vurdering}

\textbf{Mindmap og problemformulering}
Det var godt at vi startede ud med at stemme om hvad vi ville skrive om, på den måde følte alle gruppemedlemmer at de blev hørt. Vores mindmap dækkede mange interessante områder, som gav os mulighed for at skrive en god og sammenhængende initierende problemformulering, derfor har vi ikke afviget meget fra vores oprindelige problemformulering.

\textbf{Disposition}
Vi havde skrevet en meget uddybende disposition, hvilket var meget godt da det gav alle gruppemedlemmerne et overblik over hvad de skulle skrives i deres emner, samt hvad hele rapporten indeholdte. Dispositionen gjorde det nemmere for os at holde en rød tråd igennem hele projektet. 

\textbf{Tidsplan}
Det var godt at vi i starten af projektet lavede en tidsplan. Vi kom til at sætte nogle for tidlige afleveringsdatoer på, hvilket gjorde at tidsplanen ikke blev taget 100 procent seriøst. 

\textbf{Møder}
Det var positivt at have statusmøder på gruppen fra tid til anden, da det gav gruppen en ide om hvor langt vi var kommet med projektet. Der er dog plads til forbedring, når det kommer til vidensdeling.

\textbf{Logbog}
Det er godt at have en logbog, hvori vi kan se hvornår vi har lavet hvad.

\subsection{Refleksion}
`Hvorfor gik det som det gik

\textbf{Mindmap og problemformulering}
Det gik godt, fordi vi stemte om hvad vi gerne ville skrive om og det var den eneste mulighed for at gruppen kunne blive enige.

\textbf{Disposition}
Det gode ved dispositionen var at vi havde skrevet hovedpunkterne og underpunkter med detaljeret forklaring af hvad der skulle stå i de forskellige punkter. Det gode er også at vi kunne holde en rød tråd igennem hele rapporten.

\textbf{Tidsplan}
Det gode ved tidsplanen var at den gjorde så vi vidste nogenlunde hvornår de forskellige dele af rapporten skulle være færdigt. 

\textbf{Møder}
Møderne var gode, da de gav gruppen en ide om hvor langt vi var kommet, og på samme tid blev der lavet en vidensdeling blandt gruppens medlemmer. Det negative var dog at vidensdelingen ikke var uddybende nok.

\textbf{Logbog}
Logbogen er god, da det giver os mulighed for at backtracke hvornår vi har lavet de forskellige

\subsection{Syntese}
\textbf{Brainstorm og problemformulering}
Vi vil fortsætte med at stemme om emnet, så alle gruppens medlemmer føler sig hørt. Vi vil fortsætte med at lave en bred mindmap, som vi kan bruge til at lave en initierende problemformulering.

\textbf{Disposition}
Vi vil fortsætte med at skrive en uddybende disposition, da den giver et meget godt overblik over hvad der skal skrives i de enkelte punkter.

\textbf{Tidsplan}
Vi vil fortsætte med at lave en tidsplan som vi opdatere løbende, men vi vil opdatere den oftere i P2 end vi gjorde i P1. Vi vil fortsætte med at skrive ind i tidsplanen hvornår vi skal have noget brødtekst klar til at sende til vejleder, samt hvornår vi skal have møder med vejleder.

\textbf{Møder}
Vi vil fortsætte med at afholde statusmøder, men vi vil forøge mængden af vidensdeling. Vi vil prøve at afholde et statusmøde hver uge.

\textbf{Logbog}
Vi vil fortsætte med at skrive logbog, og prøve at gøre det hver gang vi mødes for at arbejde.

\section{Gruppesamarbejde}
\subsection{Beskrivelse}
\textbf{Gruppekontrakt}
I projektets startfase udarbejde vi en gruppekontrakt, som tog udgangspunkt i vores P0-gruppekontrakt, samt erfaringer fra tidligere projekter. Gruppekontrakten blev hurtigt lavet, og kan ses i bilag \ref{sec:Bilag5}. I vores gruppekontrakt har vi beskrevet at vi gerne vil afholde statusmøder. I gruppekontrakten bliver der også beskrevet, at hvis vi har aftalt mødetidspunkt og man ikke kommer, skal man give besked til gruppen. Der er også beskrevet en bødekasse som træder i kraft hvis man kommer forsent, ikke laver sine ting osv.

\textbf{Facebookgruppe}
Vi har lavet en Facebook gruppe for, at vi hurtigere kunne komme i kontakt med hinanden, både studie- og socialtrelateret. der var kun seks ud af syv som havde en Facebook konto.

\textbf{Google Drev}
Google drev blev brugt til at dele vores dokumenter, billeder og andre filer.

\textbf{Opdeling og korrekturlæsning}
Vi delte arbejdet ud ved at skrive alle emnerne op på tavlen og efterfølgende spørge gruppemedlemmerne hvad de hver især havde lyst til at skrive om. Korrekturlæsning er sket ved at man har delt et dokument der skal rettes på facebook, folk har kigget det igennem og kommenteret på det, hvorefter rettelserne er blevet indplementeret.

\subsection{Vurdering}
\textbf{Gruppekontrakt}
Det var godt at vi fokusrede meget på mødepligten, når vi havde aftalt at vi skulle møde op og arbejde på projektet. Det var også godt at der ikke var brug for at tage nogle strenge konsekvense i brug, da det kun var forsinkelser og enkelte afbud vi havde med at gøre.


\textbf{Facebookgruppe}
Det var godt at have en Facebook gruppe til at dele informationer på. Det var godt at man hurtigt kunne få svar på et problem over Facebook. Det var dårligt at den syvende mand i gruppen ikke havde en Facebook konto.


\textbf{Google Drev}
Google drev fungerede meget godt til at dele vores filer med. Det var godt at de dokumenter som blev delt ved hjælp af Google Drev havde de rigtige navne.


\textbf{Opdeling og korrekturlæsning}
Det var godt at uddele spørgsmålene efter interesse. Korrekturlæsningen var også okay, men der er dog plads til forbedringer.



\subsection{Refleksion}
\textbf{Gruppekontrakt}
Det var godt at vi fokuserede på mødepligten, da det gjorde at man var mindre tilbøjelig til at pjække eller sove over sig.


\textbf{Facebookgruppe}
Det var godt at vi havde en Facebook gruppe, da man hurtigt kunne få svar på ens spørgsmål, hvilket gjorde at man ikke skulle bruge for lang tid på at vente.


\textbf{Google Drev}
Det gik godt, da vi gav dokumenterne og filerne på Google Drev de rigtige navne, hvilket gav os et bedre overblik over hvad der præcist var delt på Google Drev.


\textbf{Opdeling og korrekturlæsning}
Det gik godt med den måde vi uddelte spørgsmålene på, da alle fik lov til at skrive om noget som de interesserede sig fo, hvilket gjorde at folk var mere engagerede til at skrive deres emne.


\subsection{Syntese}
\textbf{Gruppekontrakt}
Vi vil fortsætte med at skrive en gruppekontrakt i P2, da den påminder folk om reglerne for projektarbejdet. 

\textbf{Facebookgruppe}
Vi vil fortsætte med at bruge en Facebook gruppe til projektarbejdet, da den giver gode muligheder for vidensdelen og kontakt i gruppen. Det ville være godt hvis alle i gruppe havde en Facebook gruppe, så man ikke skal til at skrive seperate mails til ham/hende.

\textbf{Google Drev}
Vi vil fortsætte med at bruge Google Drev til fildeling, da det fungere meget godt til deling af blandt andet Latex filer og billeder.


\textbf{Opdeling og korrekturlæsning}
Vi vil fortsætte med at opdele opgaverne efter interesse.
Vi vil fortsætte med at lave korrekturlæsning på samme måde, vi skal dog gøre det oftere end vi har haft gjort i P1.


\section{Samarbejde med vejleder}
\subsection{Beskrivelse}
Vi havde i alt 5 møder med vore vejleder, Jonas Bjerg Thomsen. Før hvert vejledermøde havde vi klargjort den del af rapporten som vi gerne ville have kommentarer på, samt folk havde forberedt en række spørgsmål til vejleder. Den klargjorde del af rapporten og spørgsmålene blev sendt til vejleder et par dage før mødet. Til vejledermøderne fik vi kommentarer på rapporten og han gav svar på spørgsmålene som gruppemedlemmerne havde. Vejlederen sagde i starten at han ikke ville komme med løsninger, men derimod vil prøve at guide os hen til løsningerne. Kommunikationen med vejleder er forgået over e-mail.

\subsection{Vurdering}
Vejledermøder har generelt været gode, dog har der manglet nogle dybdegående forklaringer af nogle af de svære emner. Kommunikationen med vejleder har været god, og hans responstid var fornuftig.

\subsection{Refleksion}
Vejledermøderne var gode, da vejlederen var godt forberedt og engageret, det var på samme tid nemt at arrangere et møde med ham.

\subsection{Syntese}
Vi vil gerne have nogle mere dybdegående forklaringer, især af de svære emner.







\section{Bilag}

\subsection{Bilag 1 - Mindmap} \label{sec:Bilag1}
Fås på facebook

\begin{figure}[H] 
\centering
\includegraphics[width=0.80\textwidth]{billeder/PA1}
\caption{På figuren ses vores mindmap.}
\label{fig:Sne1}
\end{figure}


\subsection{Bilag 2 - Disposition} \label{sec:Bilag2}

Disposition:

1. Indledning


2. Problemformulering/problemafgrænsning/metodevalg


3. Beskrivels af konstruktion.

3.1: Opbygningen \textit{Formål: At læseren får et indblik i hvordan konstruktionen er opbygget, samt hvilke dimensioner den har og hvilke materialer der er blevet brugt.}

3.1.1 Konstruktionen: \textit{Hvordan er konstruktionen opbygget med henblik på gitterkonstruktion. Her vil der blandt andet vises 3D modeller/billeder af gangbroen med tvæsnit af de forskellige profiler som er anvendt. En beskrivelse af hvordan gangbroen er konstrueret vil forekomme. Vi besrkiver også gangbroens dimensioner.}

3.1.2 Materialer: \textit{Der vil ses på hvilke materialer der er brugt i gangbroen}

3.1.3 Dimensioner \textbf{Konstruktionens dimensioner.}


4. Stål og dets egenskaber. \textit{Formål: Læseren får viden omkring stålet (flydespænding). Der vil også blive beskrevet hvordan stålet vil reagere hvis det bliver påvirket ar tryk eller træk.}

4.1 Det anvendte stål: \textit{Der vil skrives om hvilke ståltyper der er anvendt i gangbroen. Her vil der illustreres med billeder af de forskellige profiler.}

4.2 Tryk- og trækstyrker (værdier): \textit{Tryk- og trækstyrker af det anvendte stål (evt. sammenligning med andre slags stål). Kræfter (tryk/træk)/vandret/lodret).}


5. Standarder og normer.

5.1 Eurocodes:

5.1.1 Krav og udregningsmetode: \textit{Find værdier til udregninern (Erhverv/gangbro) samt metoderne til at lave beregninger på konstruktionen.}

5.1.2 Nationalt anneks:

5.2 Dansk Standard

5.3 Partialkoefficienter

5.4 Bygningsreglementer


6. Beskrivelse af laster: \textit{Formål: Læser får viden omkring konstruktionens egenlast (den last som konstruktionen påvirker sig selv med, nyttelast (den last som mennesker, ting osv. påvirker konstruktionen med) og naturlaster (de laster som vind og sne påvirker konstruktionen med).}

6.1 Egenlast

6.1.1 Beregning af laster i forhold til materialer.

6.2 Nyttelast

6.3 Vindlast

6.4 Snelast


7. Statisk dokumentation \textbf{Formål: Give læser et indblik i hvordan kræfterne bliver ført igennem konstruktionen, hvilke elementer der er bærende i konstruktionen, hvilke Eurocodes der er brugt i konstruktionen, samt beregninger om blandt andet konstruktionens brudgrænser.}

7.1 Reaktioner - beregninger (bæreevne) \textbf{Vi vil finde hvilke reaktioner der opstår i konstruktionen. Ved at finde hvilke laster der er på gangbroen også regne ud hvor meget de forskellige elementer skal kunne holde til/understøtte.}

7.2 Indre kræfter: \textit{Illustration af hvordan kræfter/laster bliver ført igennem konstruktionen og finde ud af hvad for nogle elementer der er bærende/ikke bærende. Dette giver et indblik i hvordan lasterne bliver ført ud i gennem konstruktionen.}

7.3 Brudgrænser: \textit{Vi vil herunder sammenligne om de forskellige elementer har en tilstrækkelig stor nok brudgrænse for ikke at kollapse.}

8. Konklusion

9. Litteraturliste

10. Bilag

11. Appendiks




\subsection{Bilag 3 - Tidsplan} \label{sec:Bilag3}

\begin{figure}[H] 
\centering
\includegraphics[width=1.0\textwidth]{billeder/PA2}
\caption{På figuren ses vores sidst opdateret tidsplan..}
\label{fig:Sne1}
\end{figure}

\subsection{Bilag 4 - Møder} \label{sec:Bilag4}

Eksempel på vejledermøde referat.
\textbf{Vejledermøde d. 29 oktober 2015}

Ordfører: -
Referent: Kasper.
8/8 tilstedeværende (7 gruppemedlemmer og 1 vejleder).

\textbf{Referat fra mødet}

Læringsproces: I P1 procesanalyse skrives der et nyt afsnit som hedder læringsproces. Denne del består af faglig læring som er hvordan noget nyt læres. Et eksempel kunne være at der skulle læres at beregne styrkeberegninger. Dette er denne proces som skal beskrives i læringsproces. En vigtig pointe var at håndtering af konflikter i gruppen også er læring. Generel håndtering af problemer. Om læring skete individuelt eller i gruppe. Mennesker lærer forskelligt. Strukturer vores samarbejde. Hvordan gives der plads til alle i gruppen.

Roller: Der blev snakket om at man skulle afprøve forskellige roller i løbet af projektet. Ordstyrer ved vores egne møder skulle alle i gruppen prøve. Med dette kunne man evaluere ordstyreren bagefter, for at give feedback omkring forbedringer og hvad der var godt. Ud over evaluering af ordstyreren, skulle gruppen også evalueres, da der er risiko for en dårlig gruppe gør det til en dårlig ordstyrer. Her er det vigtigt at det er samspillet mellem ordstyreren og gruppen der bliver fremhævet.

I dispositionen er det ikke nok med underoverskrifter. For at give et endnu bedre overblik om den røde tråd er tydelig, skal der beskrives en kort formål og hvad hvert kapitel indeholder.




\subsection{Bilag 5 - Gruppekontrakt} \label{sec:Bilag5}


\textbf{Gruppens medlemmer}

Anders Pilgaard
Hussein Al-Ali	
Kasper Erlandsen
Oskar Kromann
Emil Faber
Mads Thomsen
Patrick Wohnberger

\textbf{Forudsætninger}

Aftalen er gældende fra d. 5. oktober 2015 til projektet er afleveret d. 18. december 2015. Den har til formål at muliggøre overholdelsen af deadlines opsat af gruppen samt blive færdig med rapporten til d. 18. december 2015. 
Alle i gruppen har fælles ansvar og mål for at aflevere og bestå dette P1 projekt. 

\textbf{Mødetid/mødepligt}

Der aftales fast mødetidspunkt, som skal overholdes af alle i gruppen. Ved afbud (f.eks. sygdom eller private forhold) skal dette ske via Facebook-gruppen eller telefon, i det personen ved, at han ikke kan komme. Der er lavet en Facebook-gruppe, som skal bruges til kommunikation og en Google Drev mappe til fildeling.

\textbf{Tidsplan}

Der laves en overordnet tidsplan for P1 projektet. To gange om ugen vurderes tidsplanen om den fastsatte deadline overholdes. Holder deadlinen ikke, laves der aftaler om flere arbejdstimer om aftenen eller i weekenden. 

\textbf{Ambition}

Rapport: Minimum 7
Eksamen: Minimum 7

\textbf{Musik}

Der må ikke høres musik under møderne, men under individuelt arbejde er det tilladt, så længe det ikke forstyrrer de andre. Hvis en person føler sig forstyrret af en anden persons musik, beder personen ham pænt om at slukke, og personen skruer ned for lyden omgående. Fuld enighed, hvis baggrundsmusik skal være gældende. 

\textbf{PC/Mobil}

Under gruppearbejdet skal der være fokus på arbejdet, og det er ikke tilladt at spille computer under gruppearbejdet. I pauserne er der ingen gældende regler for brug af PC.

\textbf{Vejleder}

Ved møde med vejleder skal alt teksten som vi sender til vejlederen kigges igennem af alle gruppemedlemmer for at sikre, at vejleder ikke skal bruge tid på at rette små dumme fejl. Gruppemedlemmerne skal, hvis det er nødvendigt, forberede spørgsmål til vejleder. Alle gruppemedlemmerne skal som udgangspunkt deltage i alle vejledermøderne. 

\textbf{Konsekvenser}

Første gang man ikke kommer og ikke har meldt afbud, tages der en snak med personen om dette. Det er ikke acceptabelt at udeblive fra aftaler uden afbud.
Hvis de andre i gruppen mener, at man ikke overholder aftaler, tages der en snak om det.

\textbf{Opsummeringsmøde}

Efter endt arbejdsdag, samles gruppen og der tages en snak om hvad der er blevet lavet, hvor meget der mangler og man vurderer, om tidsplanen overholdes. Evt. hjemmearbejde aftales.

\textbf{Morgenbrød}

En gang om ugen giver et af gruppemedlemmerne morgenmad til resten af gruppen. Hvis man glemmer at give morgenmad, bliver man tvunget til at hente det med det samme

\textbf{Bødekasse}

Hvis man kommer for sent uden aftale, skal der betales 5 kr for hvert kvarter. Det første kvarter er gratis. Hvis man glemmer at lave sine ting til det aftale tidspunkt med gruppen, betales der en bøde på 30 kr. i bødekassen. Bødekassens formål er for at overholde de relevante regler for gruppearbejde og pengene bliver brugt til en fælles bytur efter aflevering af P1 projektet. 

\subsection{Bilag 6 - Vejleders rettelser} \label{sec:Bilag6}

\begin{figure}[H] 
\centering
\includegraphics[width=0.7\textwidth]{billeder/PA3}
\caption{På figuren ses et eksempel på vejleders rettelser i vores rapport.}
\label{fig:Sne1}
\end{figure}

\subsection{Bilag 7 - Logbog} \label{sec:Bilag7}

Eksempel på logbog, eksemplet er fra torsdag d. 29 oktober 2015.

Ordfører: Oskar.
Referent: Anders.
7/7 tilstedeværende.

\textbf{Dagsplan}

1. Kl 11.15 - 12:00 midtvejsevaluering PV - Annette.
2. Beskrivelse af disposition.

\textbf{Logbog}

Vi har idag været til midtvejsevaluering hvor vi alle mødte op cirka 15 minutter før mødet. Under mødet fik vi svar og kommentarer på en masse af vores projektstyring (PV-opgaver til projektet).

Efter mødet har vi valgt at rette vores dispostion til pga. vi fik at vide vi skulle skrive hvad og hvorfor vi havde valgt de forskellige underemner i forhold til opgaven. Vi har sendt opgaven til Jonas Bjeg Thomsen (vejleder) for at få feedback over e-mail, da vi ikke har aftalt et nyt møde med ham endnu.

Efter anbefaling af vejleder har vi valgt at kommentere/vurdere rollerne som hver person har (hvordan de har klaret jobbet), vi vil prøve det nogle uger og derefter evaluere om det er noget vi skal fortsætte med eller indstille.







