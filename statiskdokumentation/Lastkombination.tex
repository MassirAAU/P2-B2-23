\chapter{Statisk dokumentation}
I det følgende kapitel vil der blive lavet en statisk dokumentation for at eftervise, at gangbroen ikke kollapser.  Først vil der komme et afsnit om lastkombinationer, som indeholder beregninger af lasterne fra kapitel \ref{laster} omregnet til regningsmæssige laster, der skal bruges til eftervisning af broens bæreevne. Herefter vil der komme et afsnit om reaktionerne for gangbroen, og derefter vil der komme et afsnit om de indre kræfter. Til sidst i kapitlet vil der komme et afsnit, hvor der vil blive eftervist, at broens profiler har en tilstrækkelig brudgrænse for at kunne bære konstruktionen med forskellige laster.


\section{Lastkombinationer} \label{Lastkombinationer}
I dette afsnit vil der blive beregnet hvilken lastkombination gangbroen skal dimensioneres efter. Dimensioneringen skal opfylde to krav:
\begin{itemize}
\item Brudgrænsetilstand (BGT).
\item Anvendelsesgrænsetilstand (AGT).
\end{itemize}

Brudgrænsetilstand og anvendelsesgrænsetilstand er to grænser, som sørger for, at en konstruktion ikke kollapser og er brugbar. Brudgrænsetilstanden er når en konstruktion eller dele af en konstruktionen er på grænsen til at kollapse, dvs. den maksimale bæreevne for konstruktionen er opnået og påvirkes den yderligere vil der opstå brud. Disse grænser findes i standarder og  skal overholdes i alle byggerier i Danmark.  Anvendelsesgrænsetilstanden er en konstruktions udbøjning, det kan for eksempel være for hvor meget en bjælke må bøje ned. Anvendelsesgrænsetilstanden er en grænsefri tilstand, som bygherren selv bestemmer. 

Lastkombinationerne findes ved at multiplicere en række partialkoefficienter på de karakteristiske laster, og derefter ligge dem sammen. Der beregnes lastkombinationer hvor alle laster på skift er dominerende, dvs. man regner med dominerende egen-, nytte-, vind- og snelast. Den dominerende last er den last, som man i beregningerne ikke reducere med partialkoefficienten $ \psi $, egenlasten reduceres heller ikke med partialkoefficienten $ \psi $. Når de forskellige lastkombinationer med forskellige dominerende laster er regnet, anvendes den største og regner videre med den, når gangbroen bæreevnen skal eftervises. Denne værdi kaldes den regningsmæssige last.

I de følgende afsnit findes lastkombinationerne indenfor brudgrænsetilstanden (BGT) og anvendelsesgrænsetilstanden (AGT). Formlerne, der er blevet brugt, er fundet i DS/EN-1990/2007.

\subsection{Brudgrænsetilstand}
I dette afsnit bestemmes brudgrænsetilstanden.  

\textbf{BGT-a:}

I denne formel antages den permanente last til at være så stor, at de resterende variable laster ikke har nogen betydning for konstruktionens bæreevne.

\begin{align*}
E_d = \gamma_{G,STR-a} \cdot K_{FI} \cdot G_k
\end{align*}
hvor:
\begin{itemize}
\item $E_d$ = regningsmæssige lastkombinationen, [$\SI{}{kN}$].
\item  $\gamma_{G,STR-a}$ = partialkoefficienten. 
\item $K_{FI}$ = partialkoefficienten for konsekvensklasse.
\item $G_k$ = egenlast, [$\SI{}{kN}$].
\end{itemize}

Eksempel på beregning af BGT-a for kundepunkt $ kl_{2} $:
\begin{align*}
E_{dkl2} = 1,2 \cdot 1,1 \cdot \SI{21,97}{kN} = \SI{29,0}{kN}
\end{align*}



\textbf{BGT-b:}
I denne formel tages der højde for, at det de variable laster som er dominerende. Det er $\psi_{0}$ som reducere lasten fra de ikke-dominerende laster i forhold til hvilken last der er dominerende. 
\begin{align*}
E_{d} = \gamma_{G,STR-b} \cdot K_{FI} \cdot G_k + \gamma_{Q_1} \cdot K_{FI} \cdot Q_{k_1} \cdot \psi_{0,1} + \gamma_{Q_2} \cdot K_{FI} \cdot Q_{k_2} \cdot \psi_{0,2} + \gamma_{Q_3} \cdot K_{FI} \cdot Q_{k_3} \cdot \psi_{0,3}
\end{align*}
hvor:
\begin{itemize}
\item $E_d$ = regningsmæssige lastkombinationen, [$\SI{}{kN}$].
\item $\gamma_{G,STR-b}$ = partialkoefficienten.
\item $K_{FI}$ = partialkoefficienten for konsekvensklasse.
\item $G_k$ = karakteristiske egenlast, [$\SI{}{kN}$].
\item $\gamma_{Q_{1}}$ = partialkoefficient.
\item $\psi_{0}$ = lastkombinationsfaktoren.
\end{itemize}

For at finde den største regningsmæssige værdi, skal man lave lastkombinationer for både dominerende sne-, vind- og nyttelast, hvorefter man tager den største af de tre værdier.

Eksempel på beregning af BGT-b for kundepunkt $ kl_{2} $:

Dominerende nyttelast:
\begin{align*}
& E_{dkl2} = 1,0 \cdot 1,1 \cdot \SI{21,97}{kN} + 1,5 \cdot 1,1 \cdot \SI{10,58}{kN} \cdot 0,6 + 1,5 \cdot 1,1 \cdot \SI{0}{kN} +  \\
& 1,5 \cdot 1,1 \cdot \SI{8,28}{kN} \cdot 0,6 = \SI{42,84}{kN}
\end{align*}


Dominerende vindlast:
\begin{align*}
& E_{dkl2} = 1,0 \cdot 1,1 \cdot \SI{21,97}{kN} + 1,5 \cdot 1,1 \cdot \SI{10,58}{kN} \cdot 0 + 1,5 \cdot 1,1 \cdot 0,8 \cdot \SI{0}{kN} + \\
& 1,5 \cdot 1,1 \cdot \SI{8,28}{kN}  = \SI{37,83}{kN}
\end{align*}


Dominerende snelast:
\begin{align*}
& E_{dkl2} = 1,0 \cdot 1,1 \cdot \SI{21,97}{kN} + 1,5 \cdot1,1 \cdot \SI{10,58}{kN} + 1,5 \cdot 1,1 \cdot \SI{0} {kN} \cdot 0,7 + \\
& 1,5 \cdot 1,1 \cdot \SI{8,28}{kN} \cdot 0,5 = \SI{48,46}{kN}
\end{align*}

Alle udregninger for lastkombinationer findes i Appendisk \ref{App:Brudgraensetilstand}. 

\subsection{Anvendelsesgrænsetilstand}
Anvendelsesgrænsetilstanden bestemmes via følgende formel:
\begin{align*}
E_d = G_k + \psi_{2.1} \cdot Q_{k_1} + \psi_{2.2} \cdot Q_2 + \psi_{2.3} \cdot Q_3
\end{align*}
\begin{align*}
E_d = egenlast + nyttelast + snelast + vindlast
\end{align*}
\begin{itemize}
\item $G_k$ = Egnelasten. 
\item $\psi_{2.1}$ = en aflæst tabelværdi i Eurocodes, \citep{Eurocode}, til den første last.
\item $Q_{k_1}$ = den første last. 
\item $\psi_{2.1}$ = aflæst i tabel i Eurocodes, \citep{Eurocode}, for anden last.
\item $Q_{k_1}$ = den anden last. 
\item $\psi_{2.1}$ = aflæst i tabel i Eurocodes, \citep{Eurocode}, for den tredje last.
\item $Q_{k_1}$ = den tredje last.
\end{itemize}

Eksempel af beregning for AGT i knudepunkt $ kl_{2} $:
\begin{align*}
E_{dkl2} = \SI{21,97}{kN} + 0,2 \cdot \SI{0}{kN} + 0 \cdot \SI{10,58}{kN} + 0 \cdot \SI{8,28}{kN} = \SI{20,89}{kN}
\end{align*}

Alle udregninger for lastkombinationer findes i Appendiks \ref{App:Anvendelsestilstanden}.

I Tabel \ref{tab:dominerendelaster} ses den totale last for hver dominerende lastkombination. Den dominerende nyttelasten er den lastkombination, der belaster gangbroen mest set over hele konstruktionen. Derfor vælges denne lastkombination til videre udregning af gangbroens dimensioner.

\begin{table}[H]
\centering
\label{tab:dominerendelaster}
\begin{tabular}{|l|c|c|c|c|}
\hline
\textbf{}  & \textbf{Dom. egenlast} & \textbf{Dom. nyttelast} & \textbf{Dom. snelast} & \textbf{Dom. vindlast} \\ \hline
Totale sum & $\SI{1260,69}{kN}$     & $\SI{2960,7}{kN}$      & $\SI{2605,59}{kN}$    & $\SI{2611,98}{kN}$     \\ \hline
\end{tabular}
\caption{Den totale last af de forskellige lastkombinationer.}
\end{table}














