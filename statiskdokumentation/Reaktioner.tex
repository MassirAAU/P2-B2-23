\section{Reaktioner} \label{kap:Reaktion}
For at kunne bestemme stangkræfterne/de indre kræfter i en gitterkonstruktion, er man nødt til at kende de ydre kræfter, samt reaktionerne på konstruktionen. Reaktionerne er den kraft konstruktionens understøtninger skal kunne optage for ikke, at konstruktionen skal kollapse.

Reaktionerne er placeret der hvor undetstøtningerne for gangbroen er. Den ene understøtning, er en simpel fast understøtning, som kaldes R$_{A}$ og har reaktionerne R$_{AV}$ og R$_{AH}$. En simpel fast understøtning kan optage kræfter i både vertikal og horisontal retning. Den anden understøtning er en simple bevægelige understøtning, som kaldes R$_{B}$ og kan kun optage kræfter i vertikal retning og derfor har den kun en reaktion R$_{BV}$. Reaktionerne kan ses på Figur \ref{fig:reaktioner}.
 
\begin{figure} [H]
\centering
\includegraphics[width=1\textwidth]{billeder/Reaktioner}
\caption{Alle knudepunkter}
\label{fig:reaktioner}
\end{figure}  
 
For at udregne reaktionerne anvendes de to ligevægtsligninger for moment og vertikal ligevægt:
\begin{align*}
\sum M = 0 \\
\sum F_{y} = 0
\end{align*}

Et moment er defineret, som kraften $F$ multipliceret med længden $a$, som er den vinkelrette afstand til angrebslinjen for $F$, også kaldes arm. For at udregne reaktionen i $k_{24}$ bruges momentligevægtsligningen i punktet $k_{13}$. Dette gøres, fordi den eneste ubekendte kraft vil være $R_{BV}$, da armens længde for $R_{AV}$ og $R_{AH}$ er nul og de vil udgå af ligningen:
\begin{align*}
\curvearrowleft\sum_{i=1}^{n} M = 0 \Rightarrow \sum_{i=1}^{12} (i-1) \m l \m (P_i + P_{12+i}) - 11 \m l \m R_{BV} = 0
\end{align*}
hvor:
\begin{itemize}
\item $n$ = antal af summationer.
\item $i$ = nummeret på knudepunktet, $k_{i}$.
\item $l$ = længden af hver sektion, 3.6 meter, [m].
\item $P$ = kraften i det pågældende knudepunkt, $k_{i}$ [kN].
\item $R_{BV}$ = den vertikale reaktion i, $k_{24}$ [kN].
\item $M$ = Moment, [$\SI{}{kNm}$].
\end{itemize}
Da det er R$_{BV}$, der skal udregnes, isoleres den i momentligevægtsligningen og der summeres for i alt 12 forskellige armlængder, som indeholder alle 24 punktbelastninger og den ubekendte reaktion. De to andre reaktioner er udgået, da deres armlængde er $\SI{0}{m}$.
\begin{align*}
R_{BV} = \sum_{i=1}^{12} \dfrac{(i-1) \m l \m (P_i + P_{12+i})}{11 \m l } =  \SI{750,85}{kN}
\end{align*}

Da R$_{BV}$ er kendt kan R$_{AV}$ bestemmes ved at tage summen af alle de vertikale kræfter, altså punktlasterne, og lægge dem sammen med de vertikale reaktionerne, R$_{AV}$ og R$_{BV}$. Punktlasterne regnes, som negative kræfter, da de går ned af i det vertikal retning:
\begin{align*}
\sum F_{y} = 0 \Longleftrightarrow \sum_{i=1}^{24} (-P_i) + R_{AV} + R_{BV} =  0
\end{align*}

R$_{AV}$ isoleres i ligningen og kraften for understøtning R$_{AV}$ bestemmes:
\begin{align*}
R_{AV} = \sum_{i=1}^{24} -(-P_i) - R_{BV} = \SI{762,81}{kN}
\end{align*}

Til sidst bestemmes den horisontale reaktion R$_{AH}$. Da der kun er en ydre kraft, som virker i den horisontal retning er ligevægtsligningen således:
\begin{align*}
\sum F_{x} =  R_{AH} =  0
\end{align*}
