\chapter{Forsøg}

BILLEDE:

%\begin{figure}[H] 
%\centering
%\includegraphics[width=0.50\textwidth]{billeder/LF1}
%\caption{Billede tekst.}
%\label{fig:LF1}
%\end{figure}


Tabeller:
Latex table generator

division
$\frac{•}{•}$
%$\SI{10}{kg/N}$



\section{Løs og fast lejring}


I dette afsnit vil beregninger til løs og fast lejring af jordarter gennemgåes. Forsøget har til formål at finde jordens relative lejringtæthed, som er et tal der vokser fra 0 til 1, når lejringstætheden går fra den løseste til den fasteste lejring.

Der er to typer af udstyr til at udfører forsøget alt efter den maksimale kornstørrelse, hvor en maksimal kornstørrelse under $\SI{5,0}{mm}$, medfører en lille cylinder, og over $\SI{5,0}{mm}$ medfører en stor cylinder. Idet der i sigteanalysen blev fastlagt en maksimal kornstørrelse på ca. $\SI{1}{mm}$,  skal forsøget tage udgangspunkt i en lille cylinder med volumen $\SI{70,0}{cm^{3}}$, og en stamper med faldhøjde på $\SI{30}{cm}$.

Den relative lejringstæthed er bestemt ved:

$I_{D}=\frac{e_{max}-e_{in situ}}{e_{max}-e_{min}}$

Hvor $e_{in situ}$ er jordartens naturlige poretal, $e_{max}$ er den løseste lejring og $e_{min}$ er den fasteste lejring. Poretallet angiver porevolumenets relative størrelse. Disse er blevet bestemt ved standardiserede laboratorieforsøg, som nu vil blive beregnet på. For en forsøgsvejleding henvises der til XXX.

\subsection{Løs lejring, $e_{max}$}

$e_{max}$ er det poretal, som bestemmes ved at lade sand løbe gennem en tragt, og  lejre sig i den pågældende cylinder.
Efter dette forsøg, kan poretallet bestemmes ved følgende udtryk:

$e=\frac{d_{s}\cdot \rho_{w}\cdot V}{W_{s}}-1$

$d_{s}$ er kornvægtfyldens relative densitet, som i dette forsøg er blevet skønnet til at være 2,65.

V er volumen af materialet, som er målt til $\SI{70,0}{cm^{3}}$

$\rho_{w}$ er vands densitet, $\SI{1}{g/cm^{3}}$

$W_{s}$ er den målte masse af materialet, efter forsøgets udførelse. Forsøget er gentaget 3 gange hvoraf følgende resultater er opnået:

\begin{table}[H]
\centering
\begin{tabular}{|l|l|l|l|l|}
\hline
Forsøg & 1     & 2     & 3     & gns.  \\ \hline
$W_{s}$ (g)    & 99,36 & 98,83 & 99,20 & 99.13 \\ \hline
\end{tabular}
\caption{Resultater fra løs lejring}
\label{my-label}
\end{table}

Derved kan poretallet for den løseste lejring $e_{max}$ bestemmes til at være:

$e_{max}=\frac{d_{s}\cdot \rho_{w}\cdot V}{W_{s}}-1=\frac{2,65\cdot \SI{1}{g/cm^{3}} \cdot \SI{70}{cm^{3}}}{\SI{99,13}{g}}-1=0.87$

\subsection{Fast lejring, $e_{min}$}

$e_{min}$ er det poretal, som findes ved den lejring, hvor sandet bliver indstampet i en cylinder efter en standerdiseret procedure. 

Poretallet kan findes ved samme udtryk for e, som kan ses i forrige afsnit. Ved dette forsøg er der ligeledes lavet 3 gentagelser, hvorved der er fundet  følgende volumener og masser:

\begin{table}[H]
\centering
\begin{tabular}{|l|l|l|l|l|}
\hline
Forsøg                   & 1      & 2      & 3      & gns.   \\ \hline
Ws (g)                   & 101,26 & 100,61 & 102,47 & 101,45 \\ \hline
V ($\SI{}{cm^{3}}$) & 61,80  & 61,45  & 62,45  & 61.90  \\ \hline
\end{tabular}
\caption{Resultater fra fast lejring}
\label{my-label}
\end{table}

Med disse værdier kan den gennemsnitlige fasteste lejring, $e_{min}$, findes som:

$e_{min}=\frac{d_{s}\cdot \rho_{w}\cdot V}{W_{s}}-1=\frac{2,65\cdot \SI{1}{g/cm^{3}} \cdot \SI{61,90}{cm^{3}}}{\SI{101,45}{g}}-1=0.62$

\subsection{Det naturlige poretal, $e_{in situ}$}

Det naturlige poretal er bestemt ved følgende udtryk:

$e_{in situ}=(1+w)\cdot\frac{d_{s}}{\gamma}\cdot \gamma_{w}-1$

w er jordens vandindhold som forholdet mellem vand- og kornmasse. Det antages for dette forsøg at vandindholdet er 0.

$\gamma_{w}$ er vands rumvægt, $\SI{10}{kN/m^{3}}$ 

$\gamma$ er materialets rumvægt. Den oprindelige volumen af den prøve, som er blevet udlejret, samt massen af sand der er brugt, er fundet ud fra triaxalforsøget. Rumvægten er da givet som:

$\gamma=\frac{kraft}{volumen}$.

Hvor kraften af materialet er:
$(\SI{1844}{g}-\SI{1444}{g})\cdot \SI{9,82}{m/s^{2}}=\SI{3,928}{N}$

(Masserne er henholdvis før og under triaxalforsøget)

Volumen er målt til $\SI{2,599}{10^{5}mm^{3}}$

Vi har altså at:

$\gamma=\frac{kraft}{volumen}=\frac{\SI{3,928}{N}}{\SI{2,599}{10^{5}mm^{3}}}=\SI{1.511}{10^{-5}N/mm^{3}}$

Nu kan det naturlige poretal bestemmes som:

$e_{in situ}=(1+w)\cdot\frac{d_{s}}{\gamma}\cdot \gamma_{w}-1=(1+0)\cdot\frac{2,65}{\SI{1.511}{10^{-5}N/mm^{3}}}\cdot \SI{10}{kN/m^{3}}-1=0,75$

Nu hvor $e_{max}$, $e_{min}$ og $e_{in situ}$ er fundet, kan den relative lejringstæthed, $I_{D}$ beregnes til:

$I_{D}=\frac{e_{max}-e_{in situ}}{e_{max}-e_{min}}=\frac{0,87-0,75}{0,87-0,62}=0,46$

